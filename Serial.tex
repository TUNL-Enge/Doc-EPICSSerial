% Created 2015-12-10 Thu 08:20
\documentclass[11pt]{article}
\usepackage[utf8]{inputenc}
\usepackage[T1]{fontenc}
\usepackage{fixltx2e}
\usepackage{graphicx}
\usepackage{longtable}
\usepackage{float}
\usepackage{wrapfig}
\usepackage{rotating}
\usepackage[normalem]{ulem}
\usepackage{amsmath}
\usepackage{textcomp}
\usepackage{marvosym}
\usepackage{wasysym}
\usepackage{amssymb}
\usepackage{hyperref}
\tolerance=1000
\date{2015-07-14}
\title{EPICS Serial Howto}
\hypersetup{
  pdfkeywords={},
  pdfsubject={},
  pdfcreator={Emacs 24.3.1 (Org mode 8.2.4)}}
\begin{document}

\maketitle

\section{ASYN Support}
\label{sec-1}
Based on \url{http://www.aps.anl.gov/epics/modules/soft/asyn/R4-1/HowToDoSerial/tutorial.html}
NOTE: \$TOP is \verb,~/project/EPICS/project/Metrolab, in this case\\
  Code can be found in \verb,~/project/EPICS/project/docs/code/Metrolab,
\emph{This may have been replaced by streamdevice support now. See below}

\subsection{Support module. This is the "driver"}
\label{sec-1-1}
\begin{enumerate}
\item Make the skeleton
\label{sec-1-1-1}
\begin{itemize}
\item Make a folder and build the skeleton code
\begin{verbatim}
mkdir Metrolab
cd Metrolab
$EPICS_SYNAPPS_BASE/support/asyn-4-26/bin/linux-x86_64/makeSupport.pl 
                  -t devGpib Metrolab
\end{verbatim}
devGpib type is best for serial communivations
\item Make some changes to configure/RELEASE
\begin{verbatim}
SNCSEQ=/data/EPICS/synApps/support/seq-2-2-1
ASYN = /home/longland/project/EPICS/synApps/support/asyn-4-26/
EPICS_BASE=/data/EPICS/base-3.15.2
\end{verbatim}
\item Make some changes to configure/CONFIG
(uncomment CROSS$_{\text{COMPILER}}$$_{\text{TARGET}}$$_{\text{ARCHS}}$ and set it blank)
\begin{verbatim}
CROSS_COMPILER_TARGET_ARCHS =
\end{verbatim}
\end{itemize}
\item Edit support code
\label{sec-1-1-2}
\begin{itemize}
\item Edit file \verb~MetrolabSup/devMetrolab.c~
\item Comment out unneeded DSET definitions (DSET$_{\text{AI}}$ must always be
included)
Note: "in" is "into EPICS". Out is "write out to device"
\begin{verbatim}
#define DSET_AI     devAiMetrolab   // analog in   - into EPICS
#define DSET_AO     devAoMetrolab   // analog out  - out to device
#define DSET_BI     devBiMetrolab   // binary in
#define DSET_BO     devBoMetrolab   // binary out
#define DSET_SI     devSiMetrolab   // string in
#define DSET_SO     devSoMetrolab   // string out
#define DSET_LI     devLiMetrolab   // long in
#define DSET_LO     devLoMetrolab   // long out
\end{verbatim}
\item Define some strings to translate BI and BO numbers
\begin{verbatim}
static char  *offOnList[] = { "Off","On" };
static struct devGpibNames   offOn = { 2,offOnList,0,1 };    

static char  *remoteList[] = { "Remote","Remote" };
static struct devGpibNames remote = { 2,remoteList,0,1 };

static char  *localList[] = { "Local","Local" };
static struct devGpibNames local = { 2,localList,0,1 };
\end{verbatim}
\item Comment out other unused string things (I uncommented
      \verb~intExtSsBmStopList~ stuff used by MBBI and MBBO)
\item Same for EFAST strings (comment out \verb~userOffOn~)
\item Now is the important part. The command array. Each element in
here is an individual command
(see \url{http://www.aps.anl.gov/epics/modules/soft/asyn/R4-1/devGpib.html#CreateInstrumentSupport})
\begin{verbatim}
/* Param 0 -- Read the displayed value */
{&DSET_SI,      // DSET of command (read a string from the device)
 GPIBREAD,      // Type (read)
 IB_Q_HIGH,     // Priority (high)
 "\x05",        // Format string to send to device 
                //            (ASCII HEX 05 - <ENQ>)
 "%s",          // String to interpret message
 0, 200,        // Error message length, return message length
 NULL,          // Conversion
 0, 0, NULL,    // P1, P2, and P3 used in conversion
 NULL,          // Name strings
 NULL           // Input end-of-string
},

/* Param 1 - Remote control */
{&DSET_BO, GPIBCMD, IB_Q_HIGH, "R", NULL, 0, 0, 
     NULL, 0, 0, NULL, &remote, NULL},

/* Param 2 - Local control */
{&DSET_BO, GPIBCMD, IB_Q_HIGH, "L", NULL, 0, 0, 
     NULL, 0, 0, NULL, &local, NULL},

/* Param 3 -- Read and convert the displayed value */
{&DSET_SI,      // DSET of command (read a string from the device)
 GPIBREAD,      // Type (read)
 IB_Q_HIGH,     // Priority (high)
 "\x05",        // Format string to send to device (ASCII HEX 05 - <ENQ>)
 "%*c%9s",      // String to interpret message (ignore a character then 9 chars)
 0, 200,        // Error message length, return message length
 NULL,          // Conversion
 0, 0, NULL,    // P1, P2, and P3 used in conversion
 NULL,          // Name strings
 NULL           // Input end-of-string
},
\end{verbatim}
\item Modify the \verb~Metrolab/devMetrolab.dbd~ to include only the
defined DSET definitions (you can comment out things using '\#')
\begin{verbatim}
device(ai,        GPIB_IO, devAiMetrolab,    "Metrolab")
device(ao,        GPIB_IO, devAoMetrolab,    "Metrolab")
device(bi,        GPIB_IO, devBiMetrolab,    "Metrolab")
device(bo,        GPIB_IO, devBoMetrolab,    "Metrolab")
device(stringin,  GPIB_IO, devSiMetrolab,    "Metrolab")
device(stringout, GPIB_IO, devSoMetrolab,    "Metrolab")
device(longin,    GPIB_IO, devLiMetrolab,    "Metrolab")
device(longout,   GPIB_IO, devLoMetrolab,    "Metrolab")

include "asyn.dbd"
\end{verbatim}
\item Edit the support module database file
\begin{enumerate}
\item Read the displayed value
\item Put in remote control
\item Put in local control
\item Read the field (in Tesla) - this record auto updates
\end{enumerate}
\begin{verbatim}
record(stringin, "$(P)$(R)NMR")
{
    field(DESC, "NMR Display Value")
    field(DTYP, "Metrolab")
    field(INP,  "#L$(L) A$(A) @0")
    field(PINI, "YES")
}

record(bo, "$(P)$(R)Remote")
{
    field(DESC, "Remote mode")
    field(DTYP, "Metrolab")
    field(OUT,  "#L$(L) A$(A) @1")
}

record(bo, "$(P)$(R)Local")
{
    field(DESC, "Local mode")
    field(DTYP, "Metrolab")
    field(OUT,  "#L$(L) A$(A) @2")
    field(PINI, "YES")
    field(VAL,  "1")    
}

record(stringin, "$(P)$(R)Field")
{
    field(DESC, "Get the field value")
    field(DTYP, "Metrolab")
    field(PINI, "YES")
    field(SCAN, ".2 second")
    field(EGU,  "Tesla")
    field(INP,  "#L$(L) A$(A) @3")
}
\end{verbatim}
\item Edit the Makefile in (\verb~MetrolabSup/Makefile~) to change the location of db file
\begin{verbatim}
DB_INSTALLS += ../devMetrolab.db
\end{verbatim}
\item Compile
\begin{verbatim}
cd MetrolabSup
make
\end{verbatim}
\item Check for \verb~libdevMetrolab.so~ in \verb~$TOP/lib~\\
      reminder that here, \verb,$TOP=~/project/EPICS/project/Metrolab,
\item Check for \verb~devMetrolab.dbd~ in \verb~$TOP/dbd~
\item Check for \verb~devMetrolab.db~ in \verb~$TOP/db~
\end{itemize}
\end{enumerate}
\subsection{Application. This is the code that runs}
\label{sec-1-2}
\begin{enumerate}
\item Make the application
\label{sec-1-2-1}
\begin{itemize}
\item Go to the \$TOP directory
\begin{verbatim}
cd ~/project/EPICS/project/Metrolab
\end{verbatim}
\item Make the base application and ioc boot directories
\begin{verbatim}
makeBaseApp.pl -t ioc Metrolab
makeBaseApp.pl -i -t ioc Metrolab
<Enter>
\end{verbatim}
\item Edit the Makefile in \verb~$TOP/MetrolabApp/src/~ to include the dbd
created previously and the asyn driver
\begin{verbatim}
# Include dbd files from all support applications:
Metrolab_DBD += devMetrolab.dbd
Metrolab_DBD += drvAsynSerialPort.dbd
\end{verbatim}
\item Do the same for the libs (before \verb~Metrolab_LIBS += $(EPICS_BASE_IOC_LIBS)~)
\begin{verbatim}
# Add all the support libraries needed by this IOC
Metrolab_LIBS += devMetrolab
Metrolab_LIBS += asyn
\end{verbatim}
\item Edit the Makefile in \verb~$TOP/iocBoot/iocMetrolab~
\begin{verbatim}
include $(EPICS_BASE)/configure/RULES.ioc
\end{verbatim}
\item Compile
\begin{verbatim}
cd ~/project/EPICS/project/Metrolab
make
\end{verbatim}
\item Make sure it exists (there should be a \verb~Metrolab~ executable)
\begin{verbatim}
ls bin/linux-x86_64/
\end{verbatim}
\end{itemize}
\item Make the startup script work!
\label{sec-1-2-2}
\begin{itemize}
\item Find the startup script
\begin{verbatim}
cd iocBoot/iocMetrolab
\end{verbatim}
\item Edit \verb~st.cmd~
\item The records need to be loaded
\begin{verbatim}
## Load record instances
dbLoadRecords("db/devMetrolab.db","P=Metrolab:,R=,L=0,A=0")
\end{verbatim}
\item Get the serial port running
\begin{verbatim}
## Serial port
drvAsynSerialPortConfigure("L0","/dev/ttyUSB0",0,0,0) 
asynSetOption("L0", -1, "baud", "19200") 
asynSetOption("L0", -1, "bits", "8") 
asynSetOption("L0", -1, "parity", "none") 
asynSetOption("L0", -1, "stop", "1") 
asynSetOption("L0", -1, "clocal", "Y") 
asynSetOption("L0", -1, "crtscts", "N")
\end{verbatim}
\item Turn on debugging
\begin{verbatim}
## Debugging
asynSetTraceMask("L0",-1,0x9) 
asynSetTraceIOMask("L0",-1,0x2)
\end{verbatim}
\item Make it executable
\begin{verbatim}
chmod 755 st.cmd
\end{verbatim}
\item run!
\begin{verbatim}
./st.cmd
\end{verbatim}
\end{itemize}
\end{enumerate}
\section{StreamDevice Support}
\label{sec-2}
Based on
\url{http://www.aps.anl.gov/epics/modules/soft/asyn/R4-24/HowToDoSerial/HowToDoSerial_StreamDevice.html}

\subsection{Create the drivers}
\label{sec-2-1}
\begin{enumerate}
\item Make the skeleton
\label{sec-2-1-1}
\begin{itemize}
\item Make a folder and build the skeleton code
\begin{verbatim}
mkdir MaxiGauge
cd MaxiGauge
$ASYN/bin/$EPICS_HOST_ARCH/makeSupport.pl -t streamSCPI MaxiGauge
\end{verbatim}
\end{itemize}

\item Make the App
\label{sec-2-1-2}
\begin{itemize}
\item Make the skeleton
\begin{verbatim}
rm -rf configure
$EPICS_BASE/bin/$EPICS_HOST_ARCH/makeBaseApp.pl -t ioc MaxiGauge
EPICS_BASE/bin/$EPICS_HOST_ARCH/makeBaseApp.pl -t ioc -i MaxiGauge
\end{verbatim}
\item Make some changes to \verb~configure/RELEASE~
\begin{verbatim}
# Asyn
ASYN = ${EPICS_SYNAPPS_BASE}/support/asyn-4-26/

# Streamdevice
STREAM = ${EPICS_SYNAPPS_BASE}/support/stream-2-6a

# EPICS_BASE usually appears last so other apps can preempt definitions
EPICS_BASE=${EPICS_ROOT}/base
\end{verbatim}
\item Edit the \verb~MaxiGaugeApp/src/Makefile~
\begin{verbatim}
# Include dbd files from all support applications:
MaxiGauge_DBD += stream.dbd
MaxiGauge_DBD += asyn.dbd
MaxiGauge_DBD += drvAsynSerialPort.dbd

# Add all the support libraries needed by this IOC
MaxiGauge_LIBS += stream asyn
\end{verbatim}
\end{itemize}

\item Setup the IOC
\label{sec-2-1-3}
\begin{itemize}
\item Edit \verb~st.cmd~
\end{itemize}
\begin{verbatim}
#!../../bin/linux-x86_64/MaxiGauge

## You may have to change MaxiGauge to something else
## everywhere it appears in this file

############################################################################### 
# Set up environment 
< envPaths

############################################################################### 
# Allow PV name prefixes and serial port name to be set from the environment 
epicsEnvSet "P" "$(P=MaxiGauge)" 
epicsEnvSet "R" "$(R=)" 

############################################################################### 
## Register all support components
cd "${TOP}"
dbLoadDatabase "dbd/MaxiGauge.dbd"
MaxiGauge_registerRecordDeviceDriver pdbbase

###############################################################################
## Serial port
drvAsynSerialPortConfigure("L0","/dev/ttyUSB0",0,0,0) 
asynSetOption("L0", -1, "baud", "19200") 
asynSetOption("L0", -1, "bits", "8") 
asynSetOption("L0", -1, "parity", "none") 
asynSetOption("L0", -1, "stop", "1") 
asynSetOption("L0", -1, "clocal", "Y") 
asynSetOption("L0", -1, "crtscts", "N")

###############################################################################
## Load record instances
dbLoadRecords("db/devMaxiGauge.db","P=$(P):,R=$(R),L=0,A=0")

###############################################################################
## Start EPICS!
cd "${TOP}/iocBoot/${IOC}"
iocInit
\end{verbatim}
\begin{itemize}
\item Make it executable
\begin{verbatim}
cd iocBoot/iocMaxiGauge
chmod 755 st.cmd
\end{verbatim}
\item Test!
\begin{verbatim}
./st.cmd
\end{verbatim}
\item Test some more
\begin{verbatim}
epics> dbl
epics> dbpf MaxiGauge:RST
\end{verbatim}
\end{itemize}
\end{enumerate}
% Emacs 24.3.1 (Org mode 8.2.4)
\end{document}